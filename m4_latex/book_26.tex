\documentclass[a4paper]{article} 

%% Language and font encodings 
\usepackage[english]{babel} 
\usepackage[utf8x]{inputenc} 
\usepackage[T1]{fontenc} 

%% Sets page size and margins 
\usepackage[a4paper,top=3cm,bottom=2cm,left=3cm,right=3cm,marginparwidth=1.75cm]{geometry} 

%% Useful packages 
\usepackage{amsmath} 
\usepackage{graphicx} 
\usepackage[colorinlistoftodos]{todonotes} 
\usepackage[colorlinks=true, allcolors=blue]{hyperref} 

\title{Your Paper} 
\author{You} 

\begin{document} 
Problem 5.40 How low can you go?
Investigate the accuracy of the approximation
pi/2
−pi/2
(cos t)
n dt ≈
	pi
n, (5.52)
for small n, including n = 1.
Problem 5.41 Closed form
To evaluate the integral
pi/2
−pi/2
(cos t)
100 dt (5.53)
in closed form, use the following steps:
a. Replace cos t with (eit + e−it)

2.
b. Use the binomial theorem to expand the 100th power.
c. Pair each term like eikt with a counterpart e−ikt; then integrate their sum
from −pi/2 to pi/2. What value or values of k produce a sum whose integral
is nonzero?
5.6 Summary and further problems
Upon meeting a complicated problem, divide it into a big part—the most
important effect—and a correction. Analyze the big part first, and worry
about the correction afterward. This successive-approximation approach,
a species of divide-and-conquer reasoning, gives results automatically
in a low-entropy form. Low-entropy expressions admit few plausible
alternatives; they are therefore memorable and comprehensible. In short,
approximate results can be more useful than exact results.
Problem 5.42 Large logarithm
What is the big part in ln(1+e2)? Give a short calculation to estimate ln(1+e2)
to within 2%.
Problem 5.43 Bacterial mutations
In an experiment described in a Caltech biology seminar in the 1990s, researchers
repeatedly irradiated a population of bacteria in order to generate mutations. In
each round of radiation, 5% of the bacteria got mutated. After 140 rounds,
roughly what fraction of bacteria were left unmutated? (The seminar speaker
gave the audience 3 s to make a guess, hardly enough time to use or even find
a calculator.)
98 5 Taking out the big part
Problem 5.44 Quadratic equations revisited
The following quadratic equation, inspired by [29], describes a very strongly
damped oscillating system.
s2 + 109s + 1 = 0. (5.54)
a. Use the quadratic formula and a standard calculator to find both roots of the
quadratic. What goes wrong and why?
b. Estimate the roots by taking out the big part. (Hint: Approximate and solve
the equation in appropriate extreme cases.) Then improve the estimates using
successive approximation.
c. What are the advantages and disadvantages of the quadratic-formula analysis
versus successive approximation?
Problem 5.45 Normal approximation to the binomial distribution
The binomial expansion
1
2 +
1
2
2n
(5.55)
contains terms of the form
f(k) ≡
 2n
n − k

2−2n, (5.56)
where k = −n...n. Each term f(k) is the probability of tossing n − k heads
(and n + k tails) in 2n coin flips; f(k) is the so-called binomial distribution
with parameters p = q = 1/2. Approximate this distribution by answering the
following questions:
a. Is f(k) an even or an odd function of k? For what k does f(k) have its
maximum?
b. Approximate f(k) when k  n and sketch f(k). Therefore, derive and explain
the normal approximation to the binomial distribution.
c. Use the normal approximation to show that the variance of this binomial
distribution is n/2.
Problem 5.46 Beta function
The following integral appears often in Bayesian inference:
f(a, b) = 1
0
xa(1 − x)
b dx, (5.57)
where f(a − 1, b − 1) is the Euler beta function. Use street-fighting methods to
conjecture functional forms for f(a, 0), f(a, a), and, finally, f(a, b). Check your
conjectures with a high-quality table of integrals or a computer-algebra system
such as Maxima.
6
Analogy
6.1 Spatial trigonometry: The bond angle in methane 99
6.2 Topology: How many regions? 103
6.3 Operators: Euler–MacLaurin summation 107
6.4 Tangent roots: A daunting transcendental sum 113
6.5 Bon voyage 121
When the going gets tough, the tough lower their standards. This idea,
the theme of the whole book, underlies the final street-fighting tool of
reasoning by analogy. Its advice is simple: Faced with a difficult problem,
construct and solve a similar but simpler problem—an analogous problem.
Practice develops fluency. The tool is introduced in spatial trigonometry
(Section 6.1); sharpened on solid geometry and topology (Section 6.2);
then applied to discrete mathematics (Section 6.3) and, in the farewell
example, to an infinite transcendental sum (Section 6.4).
6.1 Spatial trigonometry: The bond angle in methane
The first analogy comes from spatial trigonometry. In
methane (chemical formula CH4), a carbon atom sits at
the center of a regular tetrahedron, and one hydrogen
atom sits at each vertex. What is the angle between
two carbon–hydrogen bonds?
Angles in three dimensions are hard to visualize. Try, for
example, to imagine and calculate the angle between two faces of a regular
tetrahedron. Because two-dimensional angles are easy to visualize, let’s
construct and analyze an analogous planar molecule. Knowing its bond
angle might help us guess methane’s bond angle.
100 6 Analogy
Should the analogous planar molecule have four or three hydrogens?
Four hydrogens produce four bonds which, when spaced
regularly in a plane, produce two different bond angles. In
contrast, methane contains only one bond angle. Therefore,
using four hydrogens alters a crucial feature of the original
problem. The likely solution is to construct the analogous
planar molecule using only three hydrogens.
Three hydrogens arranged regularly in a plane create only
angle in methane! One data point, however, is a thin reed
on which to hang a prediction for higher dimensions. The
single data point for two dimensions (d = 2) is consistent with numerou
(60d or much else.
0 Selecting a reasonable conjecture requires gathering further
data. Easily available data comes from an even simpler yet
analogous problem: the one-dimensional, linear molecule
CH2. Its two hydrogens sit opposite one another, so the
two C–H bonds form an angle of 0 = 180.
Based on the accumulated data, what are reasonable conjectures for the threedimensional
angle 03?
d 0d
1 180
2 120
3 ?
The one-dimensional molecule eliminates the conjecture that
0d = (60d). It also suggests new conjectures—for example,
that 0d = (240 − 60d or 0d = 360/(d + 1. Testing these
conjectures is an ideal task for the method of easy cases.
The easy-cases test of higher diensions (high d) refutes the
conjecture that 0d = (240 − 60d. For high d, it predicts
implausible bond angles—namely, 0 = 0 for d = 4 and 0<0 for d>4.
Fortunately, the second suggestion, 0d = 360/(d + 1), passes the same
easy-cases test. Let’s continue to test it by evaluating its prediction for
methane—namely, 03 = 90. Imagine then a big brother of methane: a
CH6 molecule with carbon at the center of a cube and six hydrogens at the
face centers. Its small bond angle is 90. (The other bond angle is 180.)
Now remove two hydrogens to turn CH6 into CH4, evenly spreading out
the remaining four hydrogens. Reducing the crowding raises the small
bond angle above 90—and refutes the prediction that 03 = 90.
\end{document}
